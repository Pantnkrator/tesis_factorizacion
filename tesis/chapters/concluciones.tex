\clearpage
\chapter{CONCLUSIONES Y RECOMENDACIONES}

\section{CONCLUSIONES}
El algoritmo de Fermat, es el más rápido cuando la diferencia entre los factores primos es pequeña y en este campo es superior a los demás algoritmos.

Si se desea factorizar números que tengan factores primos pequeños, el algoritmo de divisiones sucesivas puede ser utilizado y tiene una complejidad muy baja en cuanto a su implementación.

El algoritmo de Pollard Rho es el algoritmo predilecto para uso general en cuanto se trate de factorizar números de uso cotidiano, por la complejidad que aporta y los factores primos encontrados.

Mientras que si se desea trabajar en el campo de la criptografía o un uso más especializado en el area de Teoría De números el mejor algoritmo en estos casos es la Criba General de Cuerpo de Números o la Criba Cuadrática, estos algoritmos requieren de una comprensión más profunda de la parte teórica y una correcta implementación para su uso adecuado.

\section{RECOMENDACIONES}
Una vez concluido el presente trabajo se sugiere la posibilidad de continuar con algunas de las líneas de investigación desarrolladas e incluso incursionar en otras relacionadas con los temas tratados, se recomienda seguir con las siguientes temáticas.

Continuar con el estudio de los algoritmos cuánticos y algoritmos híbridos para posibles avances en el area de criptografía.